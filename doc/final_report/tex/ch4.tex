% LaTeX layout by Jonas Kahler, jonas@derkahler.de
% AutoTux Final Report
% Group Tux:
% Max Enelund, Jerker Ersare, Thorsteinn D. Jörundsson,
% Jonas Kahler, Dennis Karlberg Niklas le Comte, Marco Trifance, Ivo Vryashkov
% Chapter 4 - Implementation Details
\chapter{Implementation Details}
%% Low Level Board by Jerker
\section[Low Level Board]{Low Level Board\textsuperscript{[JE]}}
%%% Neopixel Software Driver
\subsection{Neopixel Software Driver}
We wrote our own software-based bitbanging driver to interact with the neopixel
LED strips. When interacting with the neopixels, it pauses non-critical ChibiOS
interrupts momentarily while alternating the (high/low) digital state of the
output pin according to a specific timing protocol, where a certain timing of
the high and low states means 0, and another means 1. This way, a sequence of
bits are sent to the light strip.\\

\noindent
Each neopixel reads 24 bits (8 for each three colors), and then forwards any
excessive bits to the next pixel. Hence, to control 16 neopixels, we need to
send 24 * 16 bits. When all bits are sent, a pause of a certain time lets the
neopixel strip know we are finished writing, and that the corresponding LEDs
should be turned on with the desired brightness.

%%% Packet Parsing and Serial Connection Details
\subsection{Packet Parsing and Serial Connection Details}
Received bytes from the serial connection are put into a buffer. When the buffer
is large enough, packet parsing is attempted. The algorithm looks from the end
(most recent) of the buffer and steps backward until it finds the end delimiter
of a packet. When it does, it verifies that the start delimiter and size header
are also present (following the netstring standard), and then reads the values
transmitted in the packet body. The last byte in the packet body is a checksum
byte, which is the result of computing the XOR value of all other value bytes.
If the checksum byte indicates the packet is valid, the buffer is cleared,
because any older packets will not be of interest. The green LED on the STM32
board is lit up to indicate valid communication.\\

\noindent
Whenever we tried to parse the buffer but didn't find a valid packet, the orange
LED is lit on the STM32 board. This can occur either because the checksum did
not match the contents, or because several bytes (enough to expect a packet)
were received that did not constitute a packet.\\

\noindent
If no valid packets are received for a while, the red LED is lit. To make sure
the output buffer is not filled, which would cause the thread to block while
writing and could happen if the USB connection was disconnected, we make sure
not to send anything in this state. However, since the serial connection is
running in a dedicated thread, any blocking state would not in itself cause
safety problems.

%% High Level Board
\section{High Level Board}
%%% Proxy by Max, Jonas and Ivo
\subsection[Proxy]{Proxy\textsuperscript{[ME \& JK \& IV]}}
The AutoTuxProxy component extends OpenDaVINCI time-triggered module and runs in
the frequency provided by the user with the flag ``--freq=''. It consists of
four separate namespaces, each responsible for a different task - camera, proxy,
serial and containerfactory.\\

\noindent
\textsuperscript{\textbf{[ME]}}
As stated above, within this component we have the handling of the camera. In
this case a regular PC web-camera. To capture images we're using a third party
library - OpenCV (Open Source Computer Vision). This is a library aimed towards
real-time computer vision. Using this library gives us a nice and intuitive way
to communicate with the physical camera attached to the Odroid.\\
Running this module creates a SharedImage (OpenDaVINCI object) from defined
variables in the OpenDaVINCI configuration file. We then use the OpenCV library
to open (connect) to the physical camera connected to the Odroid board, and
fetch (grab), then read (retrieve, decode) the grabbed image into the
SharedImage memory location.\\
In the earlier stages of the development we stored the whole image, which we
later on changed to only copy the bottom half of the grabbed image, due to the
fact that we're only looking at that part when using the image in the
LaneFollowing module. By doing this we essentially cut the memory needed by this
module in half, if we're lucky we also saved a couple of nanoseconds when
copying the image to the memory location.\\
How ever we did have problems with the slice consumption within this module,
this was mainly because of the different lighting conditions at the test track.
As a darker room would mean that the camera needed a longer exposure time this
would also increase our slice consumption time to unacceptable values, this did
not pose as a problem when the room was better lit up.\\
To counteract this problem we tried to use OpenCV to set a reasonable static
exposure time that would work ok for a more wider area of lightning conditions.
Although this was not as easy as it would seem as OpenCV v.2 is using
\textit{Video4Linux} API version 1 and the camera properties we needed to access
is not available in versions below API v.2. Hence this is still not completely
solved.\\

\noindent
\textsuperscript{\textbf{[JK]}}
Each proxy iteration (as soon as a time slice is free), the time-triggered
module reads the newest valid packet received via USB (wrapped in a vector),
rechecks for its correct size, and sends out the data to the session. The data
the proxy sends out, is of type SensorBoardData and VehicleData. To generate the
containers we send out, two factories are used (part of the containerfactory
namespace) which return a shared pointer (to automatically get rid of these
objects after they aren't used anymore).\\
The SBDContainer factory basically takes our internal order and puts the data
from the sensors in the order OpenDaVINCI is using in its simulation.\\
The VDContainer factory is slightly more complex in it's behavior: after setting
the speed it takes all four bytes we receive from the USB for calculating the
absolute travelled path and bit shifts them together to one unsigned integer
value.\\
Note that all values for distances and speed are getting converted from
centimeter based values to meter based ones (as it is used in the high-level
code).\\
The last step is to receive the most recent containers for VehicleControl and
LightSystem data from the session. The data gets processed (steering wheel angle
from radians to degrees, speed to four different fixed values and the light
settings bit shifted into a four bit sized unsigned char) and it's checksum is
generated. Finally a vector out of these values is created which replaces the
buffer wrappers send buffer which will be sent out back to the STM32 via USB.\\

\noindent
\textsuperscript{\textbf{[IV]}}
The serial module is responsible for the serial connection between the low-level
board - the STM32 - and the high-level board - the Odroid. It reads from the usb
sensor-board data, collected  from the sensors, and writes to it control data,
evaluated by the DecisionMaker component. The communication between the two is
managed by a third-party library - libusb-1.0. Libusb is a lightweight library
that is portable and easy to use. It supports all transfer types - control,
bulk, interrupt and isochronous - and two types of transfer interfaces -
synchronous and asynchronous. Furthermore, the library is thread-safe, it does
not manage any threads by its own. The class implementing the calls for managing
the usb connection is in SerialIOImpl.cpp. When an object of that class is
created, a libusb\textunderscore context is initialized by making a call to
libusb\textunderscore init() function and passing it a reference to the context
struct. It is possible to call this function with the argument NULL , in which
case a default context is created for the user. The problem that can occur by
doing so is that - if there are multiple components in the application using
libusb to communicate with any of the usb devices - the settings for the context
might not be applicable, e.g. a part of the program needs to talk to the STM32
board but another can be talking to some other device connected to the system.
Although, our program does not have that particular problem, we decided to bind
a context to the device we use as a good practice. After the context is
initialized, the list of available devices is obtained, and the device we are
interested in is opened. To match the correct usb port, we use the STM32 vendor
and product id which can be obtained by running ‘lsusb’ in the terminal on a
Linux-based machine. Once the device has been opened, we check if the interface
we want to operate on is free, i.e. the kernel is not using it, and claim it if
that is not the case. By claiming the interface of the device, we can exchange
data through the usb serial connection.\\

\noindent
The reading from and writing to the usb are performed using the synchronous API
interface. The function to use to do the operation is
libusb\textunderscore bulk\textunderscore transfer() which takes a number of
parameters. In particular, the context to operate on, the usb endpoint, i.e. if
it is read or write, indicated by the direction bits in the endpoint address,
the buffer where to store the data, the length of the buffer, an int for the
actual bytes transferred, and the timeout for the function call. The above
mentioned function is blocking but since we do not have multiple components
using the usb and there is a dedicated thread operating with it, we decided that
is sufficient to use this API. The timeout for the call is set to 10
milliseconds for both the read and write. Libusb is also smart in a way that it
does not wait for the entire buffer to be filled. In a case that less data is
received than the size of the buffer, the read operation is terminated and the
result is returned.\\
The buffer size for reading is set to size 128. We considered it to be enough
with respect to the frequency we are running and the size of our data packets.
For writing to the usb, the length of the buffer is equivalent to the size of
the data that needs to be sent.\\

\noindent
Parsing of the received packets is done when data is appended to the
SerialBuffer object with a call to appendReceiveBuffer(), which takes a vector
with unsigned chars as elements. The function traverses through all elements,
starting from the back (most recent packet) and looks for the start and end
delimiters of a correct packet. When one is detected, the values for the
different sensors are obtained along with the checksum for this packet. The
checksum is then checked against its validity and if so, the packet is
considered valid and put in the internal buffer.

%%% Configuration Tool by Jonas
\subsection[Configuration Tool]{Configuration Tool\textsuperscript{[JK]}}
The main goal of this component was to have a tool at hand which can be used to
tweak values, like the lane-follower gains or other camera setup values, on the
run, without changing values in the source code or header files. This would've
required recompilation every time and would therefore have been very time
consuming.\\
An application with a GUI seemed optimal. Because we are running everything
through a SSH session via the console, we decided to use GNU's ncurses library,
which allowed us to create a Text-User-Interface (TUI) independent of the
terminal emulator in use.\\

\noindent
From the very beginning, we used the effc++ warnings. A lot of ``Class contains
pointer data members'' warnings arose. Research on the internet showed a
possible solution: wrapping these pointers into unique\textunderscore ptrs.
Unfortunately these wrapper pointer data types do not interwork with the ncurses
library functions, so we ended up in removing the warnings.\\

\noindent
To interact with the car's OpenDaVINCI session, a module was written which
extends the \textit{TimeTriggeredConferenceClientModule} (TTCCM). Depending on
whatever frequency is set by running the application, the ATConfigurator will
send out the setup values, as well as the selected state. To avoid lagging, a
frequency of 2 Hertz is recommended.\\
To access this session data, a Singleton class, containing of different data
members, as well as their getters and setters, was used. The TUI as well as the
TTCCM is accessing this class.\\

\noindent
To get some guidance for setting up the webcam to the correct angle, the
SharedImage is fetched from the memory and transformed into an
\textit{ASCII picture}.
Initially it was planned to use libcaca for doing that - a library, which even
supports to display these images in color. While trying to integrate libcaca
into the ATConfigurator we noticed that it spawned a separate window, instead of
being in the same window as the configurator. This was fixed by using the
libcaca ncurses driver, which unfortunately overwrote also our main ui window.\\
We ended up using an external tool to generate the picture and load it into the
application. After unsuccessfully trying out img2txt (failed because of its
usage of ANSI and UTF-8 chars), we switched over to jp2a, which generates a pure
black and white ASCII representation of the picture.\\
While using this solution some small lags can be noticed: the image file is
(over)written to whatever the frequency is set. Therefore it can happen that the
image is empty or corrupt, while jp2a tries to read it. We tried to avoid this
by using a small buffer/tmpimage within the code, but some lags can be noticed
still. Further improvements seemed not reasonable due to the late stage of the
project phase.\\

\noindent
One problem we faced while developing this application was navigating through it
with arrow, return and back keys. The problem is that the values aren't as they
are programmed in ncurses. Custom values were defined in the header file.