% LaTeX layout by Jonas Kahler, jonas@derkahler.de
% AutoTux Final Report
% Group Tux:
% Max Enelund, Jerker Ersare, Thorsteinn D. Jörundsson,
% Jonas Kahler, Dennis Karlberg Niklas le Comte, Marco Trifance, Ivo Vryashkov
% Chapter 1 - Project Organization and Milestone Planning
\chapter[Project Organization and Milestone Planning]
{Project Organization and\\ Milestone Planning}

% Product Vision
\section[Project Organization]{Project Organization\textsuperscript{[TDJ]}}
At the start of project we conducted a few meetings in order to visualize and
plan the tasks at hand. A draft of the system architecture was created which in
turn was used to delegate tasks between the groups. Parts of this draft was
later revised based on feedback we received as well as realizations made down
the line.\\\\
The team was structured into task-oriented groups in the following manner:\\
\noindent
{\tabulinesep=1.4mm
\begin{tabu}{ll}
\textbf{STM32} & \begin{tabular}[c]{@{}l@{}}Jerker Ersare\\
   Thorsteinn D. J{\"o}rundsson\end{tabular}\\
\textbf{Decision Maker, Overtaking and Parking} & \begin{tabular}[c]{@{}l@{}}
   Niklas Le Comte\\ Marco Trifance\end{tabular}\\
\textbf{Proxy and Serial} & \begin{tabular}[c]{@{}l@{}}Jonas Kahler\\
   Ivo Vryashkov\end{tabular}\\
\textbf{Image Processing and Lane Following} & \begin{tabular}[c]{@{}l@{}}
   Max Enelund\\ Dennis Karlberg\end{tabular}
\end{tabu}}
\\\\
As the project progressed and tasks were completed, group members participated
in other efforts where applicable and as needed. Among those, Jerker aided in
lane-following, Ivo conducted unit tests and Jonas created a configuration tool
used in development and vehicle testing.

% Milestones and Successfully Completed Tasks
\newpage
\section{Milestones and Successfully Completed Tasks}
We always delivered on time except the sensor recording diagram (April 12), due
to problems with hardware. We received a new ultrasonic sensor the day before
the presentation and did not manage to make recordings and diagrams in time.\\\\
\noindent
{\tabulinesep=1.4mm
\begin{tabu}{l|l}
Until Feburary 2 & \begin{tabular}[c]{@{}l@{}}Installed and learned about OD\\
   Made diagrams for presentation\end{tabular}\\
\textbf{February 02} & \textbf{Simulation Environment}\\
Until February 16 & \begin{tabular}[c]{@{}l@{}}Discussed concept and
   architecture\\Prepared report\end{tabular}\\
\textbf{February 16} & \textbf{Concept \&Architecture}\\
March 24 & \begin{tabular}[c]{@{}l@{}}Discussed the team structure, goals,\\
   tasks and responsibilities\end{tabular}\\
March 25 & \begin{tabular}[c]{@{}l@{}}Received the car, basic Arduino sketch\\
   for controlling ESC and servo\end{tabular}\\
March 28 & \begin{tabular}[c]{@{}l@{}}Sensor readings through Arduino to\\
   measure sensor ranges and angles\end{tabular}\\
\textbf{March 29} & \textbf{Revised Concept \& Architecture}\\
April 1 & STM32 reading IR sensor values\\
April 4 & \begin{tabular}[c]{@{}l@{}}STM32 reading US sensor values\\
   Basic connection between boards\end{tabular}\\
April 6 & \begin{tabular}[c]{@{}l@{}}Basic readings from the Open DaVINCI
   simulations\\ to own components\\ Camera fully working, saving images to
   SharedMemory.\end{tabular}\\
\end{tabu}}

{\tabulinesep=1.4mm
\begin{tabu}{l|l}
April 8 &  \begin{tabular}[c]{@{}l@{}}STM32 basic version of wheel encoder reading\\
Basic reading and writing from/to the usb serial connection\\
Basic parser for the data packets\\
Basic lane-following working\end{tabular}\\
April 11 & Basic parking spot detection with sensors\\
\textbf{April 12} & \begin{tabular}[c]{@{}l@{}}\textbf{Basic lane-following
   (simulation)}\\STM32 PWM output\\Stable communication between the boards\\
   Basic parallel parking done without noise\end{tabular}\\
April 13 & \begin{tabular}[c]{@{}l@{}}STM32 output based on received packets\\
   First integration between LaneFollower and Decision Maker\\
   Camera handling rewritten to use OpenCV2 API.\end{tabular}\\
April 14 & \begin{tabular}[c]{@{}l@{}}STM32 RC mode\\
   LaneFollower using new algorithm for lane-following\end{tabular}\\
April 15 & \begin{tabular}[c]{@{}l@{}}Improved data packets parser and\\
   communication between boards\end{tabular}\\
April 18 & Smoother lane-following implemented on the car\\
\textbf{April 19} & \begin{tabular}[c]{@{}l@{}}\textbf{Lane-following (car)}\\
   Serial connection running at 15 Hz\\
   The parking handles noise data\end{tabular}\\
April 22 & Initial stop-line handling \\
April 25 & \begin{tabular}[c]{@{}l@{}}STM32 LED driver and basic light logic\\
   Overtaker handles noise data\end{tabular}\\
\textbf{April 26} & \textbf{Parking \& Overtaking (Simulation)}\\
April 27 & CxxTest introduced for the serial module\\
April 28 & \begin{tabular}[c]{@{}l@{}}First working ATConfigurator version\\
   Robust stop-line detection\end{tabular}\\
\end{tabu}}

{\tabulinesep=1.4mm
\begin{tabu}{l|l}
April 29 & \begin{tabular}[c]{@{}l@{}}STM32 physical lights switch\\
   Google testing framework introduced\\
   Robust communication and data exchange between\\the two boards\\
   Parking on the real car is okey\end{tabular}\\
April 31 & Edge Detection changed to use Canny\\
\textbf{May 3} & \begin{tabular}[c]{@{}l@{}}
   \textbf{Revised lane-following \& basic parking/overtaking}
   \\Refactoring and improvement of the serial module\end{tabular}\\
May 4 & \begin{tabular}[c]{@{}l@{}}
   STM32 headlights stronger in low surrounding light\\
   Data quality check added in LaneFollower\\
   Image optimizations, reduced the memory requirements\end{tabular}\\
May 5 & Lane follower execution time measured and optimized\\
May 6 & \begin{tabular}[c]{@{}l@{}}ATConfigurator can send out selected state\\
   Overtaker refactored to work on car\end{tabular}\\
May 9 & \begin{tabular}[c]{@{}l@{}}STM32 ESC calibration mode\\
   Tests completed for the serial module\end{tabular}\\
May 10 & \begin{tabular}[c]{@{}l@{}}Lane follower road offset (car keeps to\\
   the right to keep markings in camera view)\\
   ATConfigurator shows ASCII representation of the\\current webcam image\\
   Car is working with new logic for the parking inside of\\
   the spot\\\end{tabular}\\
\textbf{May 11} & \textbf{Final presentation}\\
May 13 & Serial connection running at 20 Hz\\
\end{tabu}}

%\begin{tabular}[c]{@{}l@{}}\end{tabular}\\