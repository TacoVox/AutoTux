% LaTeX layout by Jonas Kahler, jonas@derkahler.de
% AutoTux Final Report
% Group Tux:
% Max Enelund, Jerker Ersare, Thorsteinn D. Jörundsson,
% Jonas Kahler, Dennis Karlberg Niklas le Comte, Marco Trifance, Ivo Vryashkov
% Chapter 9 - Toolchain
\chapter{Toolchain \& Odroid}
\hypertarget{totgt}{}
\label{tolbl}
%% GCC by Jonas
\section[GCC/G++]{GCC/G++\textsuperscript{[JK]}}
In our initial presentation we held in front of the class, we talked about
possibly using Clang as our compiler, due to some research we did during our C
course (where we noticed that for example software interrupts aren't as good
supported in gcc as they are in clang).\\
Due to the fact that everybody used gcc straight from the beginning and there
was not really a discussion in the group about it, we stuck with it.

%% CMake by Jonas
\section[CMake]{CMake\textsuperscript{[JK]}}
As proposed by the teacher, we are using CMake. It made the whole build process
easier and more comfortable. Also the support for CMake of all modern IDEs
allowed each member to use the IDE/development environment of his choice.\\
Another big advantage using the CMakeLists.txt is the overview we get. For
example is it easy to find out the level of warnings we are using in that
software project and the external software components (libraries, etc) we are
linking the software with.\\
Notice that we mark all INCLUDE\textunderscore DIRECTORIES as SYSTEM, to avoid
warnings coming from these external software entities.

%% Jenkins by Jonas
\section[Jenkins]{Jenkins\textsuperscript{[JK]}}
To automatically build the software components each time we push changes in
their code to a certain branch, we introduced Jenkins to our toolchain.\\
To make this possible, Jenkins is connected to our GitHub repository with a so
called webhook.\\
To ensure that also an overall build (on the Odroid) will work, we introduced an
overall build behavior based on an overall CMake file as well. The server we are
running the build on reflects the same build chain as it is installed on the
Odroid (except the compiler, because of the hardware architectures are
different). All necessary build tools and libraries are installed at the same
paths.

%% pmccabe by Jonas
\section[pmccabe]{pmccabe\textsuperscript{[JK]}}
We decided to integrate pmccabe directly into our Jenkins builds. Before each
build starts, we run the pmccabe command to show us the ten most complex
functions of the current project (see pmccabe man page).\\
To see the result, we just went to a specific build and looked at the console
output.

%% Cppcheck by Max
\section[Cppcheck]{Cppcheck\textsuperscript{[ME]}}
Analysing the code is a good way to get an overview about coverage and hidden
bugs.\\
This tool in particular is a static analysis tool. That means that the code will
be analysed in a static way, giving us an overview about static bugs as:
\begin{itemize}
   \item Out of bounds checking
   \item Memory leaks checking
   \item Null pointer references
   \item Uninitialized variables
   \item Invalid usage of STL (Standard Template Library)
   \item Unused or redundant code
   \item Obsolete or unsafe functions
\end{itemize}
Unfortunately this tool will not give us any information about code coverage as
the program is not being actively monitored while running, hence static
analysis.

%% SonarQube by Max
\section[SonarQube]{SonarQube\textsuperscript{[ME]}}
This is a open source platform for continuous inspection of code quality.\\
We use this in collaboration with Jenkins and Cppcheck, every build in Jenkins
will be inspected and the results will be uploaded to this platform where it
presents the Cppcheck results in a human readable and pleasant way.\\

\noindent
When looking at the results all warnings/failures from Cppcheck will be
transformed into a new ``Issue'' and it clearly states the bug and where it can
be found. All of the issues are also converted into Technical Debt according to
predefined rules bundled with SonarQube. This is something while it's nice to
have, we've not really been using.

\section[Operating System]{Operating System\textsuperscript{[JK]}}
To have a lightweight operating system at hand, we decided to use a Debian 8.4
\textit{Jessie} minimal image, available on the Odroid forums.\\
Most of our team members used a debian-based linux distribution before, so it
seemed to be the most reasonable choice if there would be the need for them to
work with the package manager, etc. The use of a minimal image made sure, that
we would not have any unnecessary software, like a Desktop Environment,
installed.

\section[Connection]{Connection\textsuperscript{[JK]}}
To connect to the Odroid, two of our team members were able to host a hotspot,
which the Odroid could connect to.\\
This hotspot made sure, that the Odroid had an internet connection, so updating
packages on it was possible, as well as pushing changes to the code made while
testing and integrating could be pushed to the git repository (note that these
changes were pushed with the user TacoServ).\\
We accessed the Odroid with an SSH connection subsequently.