% LaTeX layout by Jonas Kahler, jonas@derkahler.de
% AutoTux Final Report
% Group Tux:
% Max Enelund, Jerker Ersare, Thorsteinn D. Jörundsson,
% Jonas Kahler, Dennis Karlberg Niklas le Comte, Marco Trifance, Ivo Vryashkov
% Chapter 7 - Test Results
\chapter{Test Results}
%% Software Tests by Ivo
\section[Software Tests]{Software Tests\textsuperscript{[IV]}}
Google Test and Google Mock were used to perform unit testing for the serial
module to ensure packet parsing is correct in the serial buffer and that the
serial handler behaves as expected. At first, CxxTest was considered and some
unit tests were implemented in this framework. However, it was decided to switch
to the Google testing framework for its simplicity and ease of use. The CxxTest
environment did not provide a direct support to mock class object, unless the
mock objects are standard library functions. One way to solve this problem would
have been to define the mocked class functions in a separate source file and
link it at compile time. Meaning that we would have one .cpp file for the 'real'
and one for the 'mock' functionality. This would have resulted in more effort
and time spent than estimated with Google Test and Google Mock.\\

\noindent
\textbf{SerialBufferTest.cpp}\\
The tests focus mostly in the parsing of correct packets when data is received
from the usb connection. The appendReceiveBuffer() function in the serial buffer
class is tested with various fault packets, e.g. wrong delimiters, wrong
checksum, etc. Along with those, some tests check the correct behaviour of the
checksum() function.\\

\noindent
\textbf{SerialHandlerTest.cpp}\\
The tests are designed to verify the expected behaviour of the readOp() and
writeOp() functions of the serial handler class. For this purpose, we mock the
serial io interface to simulate correct or false output when performing read and
write from the serial connection. In addition, test were included for the
functions trying to connect and reconnect with the serial IO.

%% Integration Tests
\newpage
\section[Integration Tests]{Integration Tests\textsuperscript{[JK]}}
Our integration tests were mostly done manually by using the Odroid mounted on
the car itself. For some parts, like the the LaneFollower interworking with the
DecisionMaker, the OpenDaVINCI simulation was used. For integrating the proxy
component with the STM32 board, the STM32 board was connected directly to our
computers via USB to test the serial connection implementations both in the
proxy and on the STM32. The STM32 was programmed to indicate any communication
problems through it's LEDs, which helped to identify any communication problems
throughout the project.\\

\noindent
Most of the software integration went fine straight from the beginning.
Nevertheless tweaking of different values was unavoidable (for example for the
parking scenario or the overtaking). After the optimal values were found, these
were pushed to to the repository as default values (TacoServ user).\\

\noindent
While integrating other components with the DecisionMaker, we faced some issues.
When the session was up and running with one supercomponent, one proxy and the
configuration tool, the DecisionMaker component didn't always receive the
SensorBoardData and DecisionMakerMSG as it was supposed to. Sometimes it
affected both data types, sometimes just one. We ended up hard coding the state
value and restarting the DecisionMaker each time until SBD values were received.
\\
We weren't able to fix this problem. Rechecking with the teacher confirmed that
we are using the receiving mechanisms in a correct way. We suspect it may have
to do with processing time issues or maybe the Linux environment on our Odroid,
and we would have investigated this further if we had more time.
