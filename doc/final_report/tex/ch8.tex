% LaTeX layout by Jonas Kahler, jonas@derkahler.de
% AutoTux Final Report
% Group Tux:
% Max Enelund, Jerker Ersare, Thorsteinn D. Jörundsson,
% Jonas Kahler, Dennis Karlberg Niklas le Comte, Marco Trifance, Ivo Vryashkov
% Chapter 8 - Project Retrospective
\chapter{Project Retrospective}
\textsuperscript{\textbf{[JK]}}
Personally I think the group size was way to big considering the scope of this
project. Due to these circumstances we did some additional stuff like the
ATConfigurator tool, which was an awesome experience to work at. Especially
learning about ncurses will be helpful for further UNIX/Linux developing.\\
When it comes to the Odroid board, we noticed that it would’ve been good to have
a faster micro SD card available, because of the very low read and write speeds
we had to and from it.\\
Another improvement would’ve been a better WiFi module. Both of these
limitations led to a very slow connection to Odroid (see section ~\ref{tolbl}:
\hyperlink{totgt}{Toolchain \& Odroid}). It could also be that the Odroid is
already a bit outdated. A switch to an RPI v3 for the next students could be
helpful.\\
Group wise, I would follow a better software process the next time, due to the
fact that it helps with the communication and tasks splitting in the group. This
time it felt sometimes a bit unstructured and improvised.\\
But overall: a fun project and I think a lot of our members will be up for the
Carolo Cup if asked.\\

\noindent
\textsuperscript{\textbf{[IV]}}
Things that went well include the serial connection and the use of several
testing frameworks. For the usb communication, it was a good experience to work
with the libusb-1.0 library. It is well developed and tested, and provides a lot
of freedom to the programmer as to how to implement a usb connection. The
supported synchronous and asynchronous API is fun to work with and also presents
a challenge, especially the latter one. At first, the usb connection was
realized using asynchronous reading and writing, and therefore manually handling
the callback function from the io operations. However, due to the lack of
multiple devices and a dedicated thread to the execution of the io, as well as
the similarity of the two implementations, a switch to using the synchronous API
was made. Therefore, reducing the complexity of our code. In addition, learning
how to use CxxTest and Google test framework was a useful and valuable
experience.\\

\noindent
Things that we could have done better is to perhaps put more effort into the
design of the entire system. For example, up until the very end of the project,
we had a problem when starting up the different components of our application.
Some components, like the LaneFollower, would start normally and receive values
from the Proxy whilst others, like the DecisionMaker, would not receive any
values at all. That led to multiple restarts of the failed component until it
connected properly and started receiving values.\\

\noindent
\textsuperscript{\textbf{[TDJ]}}
We felt that the group size was too large for this project. The project
progressed at a very acceptable rate and deadlines were met, yet tasks were at
time too saturated with manpower.\\
Choosing the STM32 over the Arduino opened a lot of doors, but proved to be a
bit complex to get up and running.\\
A number of hardware components were in a poor condition when we received the
car. The side-rear infrared sensor gave wildly fluctuating values, and the front
right ultrasonic sensor simply did not work on the STM32 despite intensive
troubleshooting.\\
It would’ve been interesting to work with a properly functioning wheel encoder
as well. Aligning it on its axis was hard, as the wheel did not conform to the
requirements set by the encoder.\\

\noindent
\textsuperscript{\textbf{[JE]}}
In my opinion, the only really challenging part in organizational terms in this
project has been how to make sure the developers of the respective behavioural
components (lane following, overtaking and parking) have had enough time with
the car to test their components. We were warned for this and relatively
prepared but still it's hard to imagine the extent of this challenge
beforehand.\\

\noindent
It would have been interesting to have better hardware, such as better infrared
sensors with longer range, a wheel encoder suited for the wheel or other way of
measuring distance, and a faster camera (in terms of exposure time in darker
conditions). That said, overall I'm happy to have learned a lot about embedded
systems programming and interacting with hardware.\\

\noindent
\textsuperscript{\textbf{[MT]}}
I feel like I could use some extra days to test the overtaking on the real car.
Working with the simulator was fun and useful at the same time, but the
transition to the real world required some major changes that could have been
optimized with some additional time.\\

\noindent
\textsuperscript{\textbf{[ME]}}
Personally I felt that the group worked very well, we managed to split the
different parts of the complete project down to smaller modules and assigned
these modules to smaller groups.\\
Although the group size was a minor problem when it came to actually
implementing parts on the car, as we needed to troubleshoot while running things
on the car it became hard timewise, also adding to this was the limited amount
of power on time because of only having one battery.\\
I also felt that the Odroid board was not really up to par with the rest of the
hardware, it was sluggish and slow when it came to working on the board. If this
was because of the WiFi module or something else I don’t know, but being slower
than a Raspberry Pi One while having better hardware I find a bit weird.\\
To sum, fun project! Definitely continue doing this, maybe have some extra
hardware available for failures. Also, make sure that the cars keeps a
consistent quality amongst groups. Several cars had problems because of poorly
executed soldering, building etc.\\

\noindent
\textsuperscript{\textbf{[NLC]}}
I feel that the group worked well in this project, but it was to big groups to
work with. Sometimes it felt like some people had nothing to do at some points.
Me myself had a lot of problems with the OpenDaVINCI and the virtual machine in
the beginning, when code that had worked one moment ago stopped working and
getting errors from the OpenDaVINCI. I reinstalled everything around five times
before choosing to use another laptop with linux installed on it. After that it
was fun to be able to work and create the parking scenario. The biggest group
problem we had was that a lot of people wanted to test on the car on the same
time and finding time for everyone wasn’t easy, and that we had a limit on the
battery made it even harder. For me I would have liked to have even more time
for testing the parking, even though the end result was good.\\
But overall it was a good project that we as a group made and we can be happy
about how the car came to be.\\

\noindent
\textsuperscript{\textbf{[DK]}}
Personally I enjoyed this project a lot more than I anticipated at the start.
The fact that the project was so hardware adjacent turned me off a bit seeing as
I enjoy high-level programming a lot more. When I realised that we were allowed
to split up the work so that people worked on completely separate parts however,
I quickly jumped on to work on the high-level board, more specifically the Lane
Following and Detection part. Working on this part of the project was what made
it enjoyable for me.\\
The group co-operation worked very well. At the start of the project we split
the group into smaller teams, even though we ended up co-operating a lot between
teams towards the end of the project, we did stick to our initial teams
throughout most of the development process. I would say that I think that the
size of the group was a bit on the large side. We have worked in groups of about
six members before, I think groups of that size works a bit better in terms of
finding sizeable tasks for everyone in the group.\\
OpenDaVINCI was a big hassle at times, especially the installation process and
the generation of data structures. Just the fact that we had to recompile
OpenDaVINCI on all machines including the Odroid every time we wanted to add a
new data structure felt like our time was a bit wasted. I do not know if there
is a better way to do this, but it can be something worth looking into for
further projects like this. Maybe really stress the generated data types early
in the project.\\
As I said though, the project was a lot more fun than I expected and I am happy
with how it turned out.
